\documentclass{article}


\usepackage{PRIMEarxiv}

\usepackage[utf8]{inputenc} % allow utf-8 input
\usepackage[T1]{fontenc}    % use 8-bit T1 fonts
\usepackage{hyperref}       % hyperlinks
\usepackage{url}            % simple URL typesetting
\usepackage{booktabs}       % professional-quality tables
\usepackage{amsfonts}       % blackboard math symbols
\usepackage{nicefrac}       % compact symbols for 1/2, etc.
\usepackage{microtype}      % microtypography
\usepackage{siunitx}
\usepackage{titling}
\usepackage{hyperref}
\usepackage{ragged2e}
\sisetup{per-mode=symbol}
\usepackage{lipsum}
\usepackage{parskip}
\usepackage{tabularx}
\usepackage{fancyhdr}       % header
\usepackage{graphicx}       % graphics
\graphicspath{{images/}}     % organize your images and other figures under media/ folder

%Header
\pagestyle{fancy}
\thispagestyle{empty}
\rhead{ \textit{ }} 

% Update your Headers here
\fancyhead[LO]{Aim Is All You Need}
% \fancyhead[RE]{Firstauthor and Secondauthor} % Firstauthor et al. if more than 2 - must use \documentclass[twoside]{article}

% Define the subtitle command
\pretitle{%
  \begin{center}
  \LARGE
}
\posttitle{%
  \par\vskip0.5em%
  \begin{large}
  \textsl{\thesubtitle}%  This prints the subtitle
  \end{large}\par\end{center}%
  \vskip1.5em
}
\newcommand{\subtitle}[1]{\def\thesubtitle{#1}}
  
%% Title
\title{Aim Is All You Need}
\subtitle{A Conceptual Introduction to Externally Pulsed Propulsion}

\author{
  Seth Katz \\
  \texttt{katzseth22202@gmail.com} \\
}

\newcolumntype{L}{>{\RaggedRight\arraybackslash}X}
\newcolumntype{C}{>{\RaggedRight\arraybackslash}c}

\begin{document}
\maketitle

\begin{abstract}\label{sec:abstract}
 In 2017, Google Research published \textit{Attention is All You Need} \cite{vaswani2023attentionneed}.  Their paper introduced the Transformer, which let neural networks capture long range dependencies.   Just a few years later, OpenAI developed tools like ChatGPT \cite{chatgpt} that resemble hypothetical early prototypes of the computers in Star Trek \cite{startrek}.

But here's the bittersweet truth:  While our screens flicker with progress, the tangible realms of space, energy and paleontology remain comparatively stagnant.

\textbf{Dude, Where's my spaceeship?}

This paper's goal is to enable our progress in physical realms to catch up with our progress online.  Our journey requires applying a single unifying idea that is much simpler than attention - aim.   \begin{figure}[htpb]
    \centering
    \includegraphics[width=0.5\linewidth]{images/Starship_Impact_ellipse.png}
    \caption{Balloons from a first rocket (not shown) crash into a second target rocket and provide propulsion \cite{earth_image}}
    \label{fig:balloon_impact}
\end{figure}

Consider two rockets.   The first deploys a series of fast moving low density balloons with miniaturized rockets capable of small navigation adjustments.  As shown in \autoref{fig:balloon_impact} these balloons precisely follow a path to sequentially crash  into a separate  target rocket.   These precise collisions deliver high-density jolts of pulsed energy and momentum to the target vehicle, enabling a surprisingly broad range of groundbreaking applications (See \autoref{tab:applications}).    

Paralleling their role in consumer AI, neural networks offer a promising avenue to extend current CubeSat formation flying algorithms with the precise control required for this externally pulsed propulsion.

\begin{table}[!htpb]
    \centering
    \caption{A Checklist of Grand Challenges Externally Pulsed Propulsion Can Solve}
    \label{tab:applications}
    \begin{tabularx}{\textwidth}{|c|L|}\hline 
        \textbf{Safer Satellite Launch} & We'll launch expensive satellites and astronauts without strapping them to fragile extremely high propellant mass fraction rockets.   See \autoref{sec:starship_safelaunch}. \\\hline
        \textbf{Suborbital Transit} & We'll create a viable suborbital travel vehicle allowing passengers to take off from normal airport runways and reach anywhere in the world in less than 2 hours. Noise pollution near population centers will be no worse than it is with conventional aircraft.  See \autoref{sec:200_mile_high}.\\\hline
        \textbf{Rocket Revolution} & We'll end the tyranny of the rocket equation. We'll still use small rockets, but giant rockets with high propellant mass fraction will no longer be needed to reach orbit. See \autoref{sec:no_isru_rocket}\\\hline
        \textbf{Lunar Lift-off} & Launching from the moon may still require volatiles, but not the more difficult task of making and storing high performance rocket fuel. \autoref{sec:lunar_rockets_no_fuel}\\ \hline
        \textbf{Jurassic Dark} & As a side effect of our advances, we'll create the field of lunar paleontology and discover concrete evidence for how life originated on Earth. We'll build a genetic record of extinct species from ancient geological periods, like the dinosaurs. See \autoref{sec:jurassic_dark} \\\hline
        \textbf{Straw Ways to Heaven} & We can construct terrestrial megastructures, extending from the ground to the edge of space, without relying on advanced magnetic technologies such as Lofstrom Launch Loops \cite{lofstrom_loop}. A particularly ambitious yet beneficial example might be a vacuum tube connecting Earth and space, a "Straw Way to Heaven." See \autoref{sec:straw_way_to_heaven}.\\\hline
        \textbf{Carbon Cancelled} & We will solve our energy problems with carbon negative fuel that absorbs the carbon dioxide produced by industry, all while using minimal land and resources.  See \autoref{sec:death_star}.\\\hline
        \textbf{Moon mining} & We'll develop in-situ resource utilization (ISRU) technology, first on our moon (See \autoref{sec:lunar_mining}) and then on icy moons like Saturn's moon Phoebe (See \autoref{sec:greedy_phoebe})\\\hline
        \textbf{Cosmic Commutes} & We'll  build fusion powered spaceships with earth-like gravity produced by constant acceleration and deceleration between destinations.  \autoref{sec:epstein_drives}\\\hline
    \end{tabularx}
\end{table}

Note, this paper summarizes some of the ideas in the blog ``Aim Is All You Need"\cite{aim2024}.
\end{abstract}

\section{Introduction}
Conceived in the 1950s and later popularized by \textit{The Three-Body Problem} \cite{liu2014three}, Project Orion remains one of the most compelling rockets ever proposed. It uniquely offers specific impulses surpassing those of electric propulsion while delivering thrust levels on par with chemical rockets \cite{projorion}.   Orion would propel a spacecraft by directing hypervelocity plasma from nuclear explosions onto a pusher plate. Despite its theoretical promise, Orion faced insurmountable challenges related to political feasibility, radioactive fallout, and the impractical mass requirements stemming from the high minimum yield of nuclear explosions. Nevertheless, Orion conceptually validated the physics of hypervelocity pulsed propulsion.   

Our approach replaces nuclear bombs with precisely aimed, hypervelocity gas balloons sequentially impacting a rocket's pusher plate. These balloons can be downscaled to practical sizes.  Unlike nuclear explosions, small hypervelocity gas impacts could be efficiently contained within a pulsed propulsion chamber.

Leveraging gravity assists and the Oberth effect \cite{oberth_effect}, hypervelocity balloon impacts could achieve high energy densities and specific impulses.  Recent cube sat formation flying and neural network advancements make the extremely precise navigation for this externally pulsed propulsion viable.  This paper presents a preliminary "back of the envelope" analysis, intended to establish the concept's potential. More rigorous, high fidelity analyses are reserved for future work.

\section{Groundwork and Prior Work for Externally Pulsed Propulsion}
\subsection{Navigation for Externally Pulsed Propulsion}
Millimeter accuracy CubeSat formation flying is rapidly advancing. The European Space Agency's (ESA) Proba-3 mission recently demonstrated this capability by creating artificial solar eclipses with a precise two CubeSat formation \cite{esa_proba_3}. The Stanford Space Rendezvous Laboratory's VISORS mission, nearing flight readiness (as of June 2025), will further showcase precise formation flying with an increased number of spacecraft \cite{guffanti2023autonomous}.  

Our hypervelocity balloons employ formation flying algorithms for precise positioning.  Unlike Proba-3 and VISORS, our formation lasts a single orbit. This means our balloon micro-thrusters face less demanding fuel efficiency requirements.

We want to maximize gas mass and minimize the solid mass for mini-thrusters and electronics. To achieve this, a few coordinator CubeSats with enhanced computing power will measure balloon positions and relay adjustment instructions.  As shown in \autoref{fig:coordinator-nodes}, these heavier coordinator nodes fly along with the balloons but do not impact the target rocket.  We keep the electronics on the balloons as simple and low mass as possible.   Each balloon, carrying perhaps a maximum of 100 grams of solid mass for sensors, power, mini-rockets, microcontrollers, and low-bandwidth communication with coordinator nodes, will detach its solid components shortly before hypervelocity rocket impact so they don't damage the target rocket.    Alternatively, the solids can fly through a small aperture in the pusher plate. Given that thousands of balloons are needed per mission, mass production will reduce costs.    

\begin{figure}[htpb]
    \centering
    \includegraphics[width=0.5\linewidth]{images/Coordinator Nodes.png}
    \caption{Coordinator nodes that do not impact the spacecraft handle more complex measurement and computation.  Balloon solids can detach before spacecraft impact, or fly through a small aperture in the plate.}
    \label{fig:coordinator-nodes}
\end{figure}

Equipped with a high performance rocket based reaction control system, the target craft will precisely navigate and adjust its position to intercept each balloon in the formation.   Given the large number of balloons and high confidence in their relative positions, algorithms like Kalman filtering should enable precise state estimation for interceptions. After each balloon impacts the rocket, we'll need to rapidly adjust for any disparities between predicted and measured propulsion. This requires quickly solving for the necessary thrust adjustments to intercept the next balloon, all while adhering to the constraints on our rocket's performance. With thousands of balloons, the rocket's system must also be robust to occasional balloon failures caused by micrometeorite impacts or manufacturing defects. Neural networks show significant promise here because they can provide controllers with accurate initial guesses for trajectories \cite{guffanti2024transformerstrajectoryoptimizationapplication} and constraints \cite{briden_constraint}.   These good initial guesses produce fast convergence without compromising control guarantees.   Additionally, reinforcement learning shows promise for developing fast space trajectory recovery algorithms for anomalies like lost balloons \ \cite{zavoli2021reinforcement}.   This powerful combination of neural networks for rapid rocket adjustments and accurate formation flying can effectively keep the rocket on course for the next balloon interception.
\subsection{Mass Fraction of Balloon to Rocket Mass}
For externally pulsed propulsion to be a viable option, the ratio of balloon propulsion mass to rocket mass must be low for relevant rocket velocity changes. We develop a closed-form approximation for this ratio ( \autoref{eq:balloon_ratio} in {\autoref{sec:balloon_ratio_approximation}). Since real collisions are not perfectly elastic, our approximation includes an energy loss "fudge factor," \(e\).  Justification for a high \(e\) comes from Project Orion's findings \cite{orion_reflections}.   The Orion team found that pusher plate collisions could be opaque. This opacity implied minimal kinetic energy loss to pusher plate heating, resulting in a more elastic impact. Additionally, a curved, roughly parabolic pusher plate would produce a  highly collimated gas reflection. Therefore, we select a relatively high  \(e=0.8\).   

Relevant mass ratios and mission scenarios are summarized in \autoref{tab:mass_scenarios}. This table clearly shows that externally pulsed propulsion can lift significant mass into orbit. For instance, if a reusable rocket like SpaceX Starship \cite{starship} lifts 25 tons of balloons into a trans-lunar orbit, they can propel a 32 ton target craft into low Earth orbit. This exceeds the Space Shuttle's maximum capacity \cite{space_shuttle_program} or the zero fuel mass of smaller regional jets like the Embraer E170 \cite{embraer_e170}.

\begin{table}[!htpb] % tabularx should be inside a table environment
    \centering
    \caption{Mass Ratio of balloon to rocket with fudge factor \(e=0.8\)}
    \label{tab:mass_scenarios}
    \begin{tabularx}{\textwidth}{|p{4em}|p{4em}|p{4em}|p{6em}|L|}\hline
        \textbf{Rocket
        Final
        Velocity
        (\SI{}{\km\per\second})} & \textbf{Balloon
        Velocity (\SI{}{\km\per\second})} & \textbf{Rocket
        Initial
        Velocity (\SI{}{\km\per\second})} & \textbf{Rocket/Balloon
        Mass
        Fraction} & \textbf{Scenario} \\\hline
        7.784 & 10.916 & 0 & 1.28 & Eccentric balloons with apogee at lunar distance pushes rocket to minimal low Earth orbit\\\hline
        0 & 7.784 & 7.784 & 2.308 & Decelerate intercity rocket for powered reentry with retrograde balloons in low orbit \\\hline
        0 & 10.916 & 7.784 & 2.97 & Decelerate intercity rocket for powered reentry with retrograde balloons from lunar orbit \\\hline
        21.518 & 69.277 & 0 & 4.30 & Balloons approach Earth from Jupiter retrograde Hohmann trajectory and push the object to escape velocity and then to a periapsis near Parker Space probe \\\hline
        42.673 & 69.277 & 0 & 1.671 & Balloons approach Earth from Jupiter retrograde Hohmann trajectory and push the object to escape velocity and then to a periapsis near Parker Space probe but in a retrograde orbit around the Sun \\\hline
        3 & 6.421 & 0 & 2.54 & Launch balloons from the moon to trans-lunar injection orbit with low periapsis on Earth \\\hline
        0 & 2.58 & 2.38 & 2.17 & Decelerate trans-lunar payloads to land on the moon \\\hline
    \end{tabularx}
\end{table}

\subsection{Handling Space Debris}
The solid components of each balloon risk becoming space debris. To avoid this, we can intercept the balloons prior to their orbital periapsis.  For example, a balloon interception at a 200 kilometer altitude likely follows a predictable trajectory with negligible atmospheric drag.   If the balloon periapsis is near 100 kilometers, this likely deorbits any solid components.   They ideally then burn up or fall over remote oceans.   Two other techniques prevent long lasting space debris:

\begin{itemize}
\item The J2 perturbation induces a changing argument of periapsis. Any balloon orbit designed to avoid specific critical angles will eventually have a periapsis over the equator.  Since the Earth is wider at the equator, periapsis will be at a lower atmospheric altitude with much more drag.
\item If our balloon interceptions occur at night, subsequent daytime periapsis will be over a warmer atmosphere with higher drag at the same altitude.
\end{itemize}

\subsection{Balloon Deployment Details for Low Earth Orbit Scenario}
Let's consider the complexities of launching target rockets into Low Earth Orbit (LEO) with propulsion from balloons deployed in highly eccentric Earth orbits.

It's crucial to acknowledge that these balloons will not follow strictly identical Keplerian orbits. As the target rocket accelerates, the optimal balloon interception point moves along with it, so that successive balloons have slightly different orbital elements.
For low-density balloons, solar radiation pressure will significantly distort their orbits. To simplify orbit planning and maintain consistent  effects across the fleet, it's ideal for each balloon to remain outside the shadow of the others. This may necessitate slight variations in orbital parameters, such as inclination, for nearby balloons. The micro thrusters on each balloon  make corrections for unmodelled perturbations.  We still ensure each balloon achieves the correct interception position with the target rocket.   

Fortunately, strategically deploying our balloons near apoapsis can minimize the $\Delta v$ required for these inclination changes. Apoapsis balloon deployment also minimizes the $\Delta v$ for appropriate balloon spacing. The small $\Delta v$ changes required at apoapsis should enable practical designs for a very precise mechanical balloon launcher onboard our balloon deployment rocket.

Low density gas would consume too much volume in a balloon deployment vessel.   Instead, the balloons can be frozen and thaw once deployed from the rocket.  
\section{To The Moon by Water Balloon:  The Business Case for Externally Pulsed Propulsion}   
\subsection{Making Starship's Excess Capacity Useful To Satellite Customers} \label{sec:starship_safelaunch}
SpaceX's Starship may be the first fully reusable rocket. At scale, this could dramatically reduce space flight costs.   However, the rocket's large payload size is probably too big for today's space industry needs. For example, in \textit{The New York Times}
\begin{quote}
Carissa Christensen, the chief executive of Bryce Tech, an analytics firm that tracks the launch market, says launching Starship frequently will be key to closing SpaceX’s business case, but finding customers to fill the rocket’s giant payload capacity will be challenging. 
"Starship's payload capacity is huge; it's very, very big, and there aren't that many commercial uses today for a rocket that big," she said.  "Maybe it'll be so cheap that it makes sense to launch satellites on it if its not full or near full."  \cite{nyt_starship_size}
\end{quote}
Although many satellites might be too small to fill up Starship, they can still be very expensive to build and test. For an extreme example, the James Webb Telescope \cite{james_webb_space_telescope} cost \$9.7 billion dollars \cite{jwst_cost} to build. \textit{McKinsey Consulting} argues 
\begin{quote}
Safety and reliability will continue to be overarching concerns, suggesting excellent execution will be table stakes for a competitive launch company. \cite{mckinsey_reliability}
\end{quote}
Once rockets start ferrying astronauts, reliability becomes even more emphatically non-negotiable. Fully stacked, Starship weighs \SI{5}{\kilo\tonne} at launch \cite{starship}, yet its empty mass is only \SI{300}{\tonne}. That’s an awfully thin soda can strapped to a \SI{5}{\kilo\tonne} bomb that we're entrusting with billion-dollar satellites or human passengers.

By contrast, a suborbital rocket that merely reaches 200 kilometers in altitude might have a propellant mass fraction under 50 percent. This greater structural margin allows for more robust, reliable engines and extensive payload protection systems. Of course, such a suborbital rocket cannot reach orbit on its own.  It needs Starship to "push" it with externally pulsed propulsion.
\begin{figure}[htpb]
    \centering
    \includegraphics[width=0.5\linewidth]{images/suborbital_push_cartoon.png}
    \caption{A fun and of course very technically accurate cartoon about a suborbital rocket's $\Delta v$ limitations}
    \label{fig:suborbital-cartoon}
\end{figure}



Starship launches balloons for externally pulsed propulsion into a highly eccentric orbit around Earth, with a low 200-kilometer perigee and a high apogee. These balloons move near Earth’s escape velocity at perigee. The suborbital rocket intercepts the balloons with its pusher plate and rides their momentum pulses to low Earth orbit. While this approach reduces payload capacity compared to a single rocket system, most satellites are unable to take advantage of Starship’s enormous payload volume anyway.

\subsection{The 200 Mile High Club} \label{sec:200_mile_high}
SpaceX proposes direct Earth-to-Earth travel \cite{earth_to_earth}, ballistically launching passengers directly between cities with Starship from offshore platforms. This idea faces significant challenges:
\begin{itemize}
\item Marine environmental impact: Noise pollution from sea launches could severely harm marine ecosystems, including dolphins and whales.
\item Noise over populated areas: Starship's descent would generate loud sonic booms over cities.
\item Logistical inefficiency: Offshore launch platforms, by necessity located far from land, would add hours to the journey for boarding and departure, undermining the benefit of rapid travel.
\end{itemize}
A suborbital rocket plane offers a more practical alternative. To reach orbit, the suborbital rocket requires propulsion balloons launched by the Starship.   Starship can be launched from remote locations where noise pollution is acceptable.  Our suborbital rocket plane could take off from conventional urban airports using standard aircraft engines, then switch to rocket propulsion at high altitudes over remote areas to "skip" above the atmosphere and intercept Starship's propulsion balloons.   While normal airports can handle kerosene or methane fuel, most lack the infrastructure for cryogenic liquid oxygen.

However, two solutions address the liquid oxygen challenge:
\begin{itemize}
\item \textbf{Air-breathing Scramjets (Optimistic Scenario):} If economical passenger scramjets become feasible, rockets could accelerate to around Mach 7 before a steep climb, eliminating the need for liquid oxygen. This would also reduce propellant mass. Unfortunately, the extreme stress on the airframe makes suborbital rockets more economically viable for the foreseeable future.
\item \textbf{Mid-air oxygen refueling:} Suborbital rocket planes could receive liquid oxygen from specialized aircraft launched from airports equipped with liquid oxygen storage. This avoids retrofitting every urban airport, requiring only a single, strategically located airport near major destinations.   Mid-air refueling also reduces our rocket plane's takeoff weight.
\end{itemize}

A formation of propulsion balloons launched in an eccentric Keplerian orbit could only propel the  rocket plane on trajectories that bisect the Earth. Since most urban destinations would not align with such trajectories, the rocket plane needs to fire an adjustment burn at least once to reach the destination. Alternatively, a second set of propulsion balloons could intercept the rocket plane mid-flight and adjusting it's trajectory. Furthermore, to avoid a sonic boom upon descent, the plane must decelerate before atmospheric reentry. This deceleration could be provided by yet another set of propulsion balloons that were traveling in a retrograde orbit relative to the plane. Notably, these reentry balloons could be in a circular low Earth orbit rather than a high eccentricity orbit, significantly reducing the mass required.

\section{I Love ISRU.  Cheaper Externally Pulsed Propulsion \textit{Without} Giant Reusable Rockets}
For externally pulsed propulsion to achieve genuine economic competitiveness, the cost of launching its propulsion balloons must be exceptionally low.  The rocket plane must be robust enough to guarantee on-time takeoffs and avoid missing its initial balloon rendezvous.  Rocket acceleration must be limited for infants and other susceptible passengers, further increasing costs and $\Delta v$.  Even highly reusable rockets may prove too expensive, limiting this technology primarily to the very high-end private luxury aviation market. While luxury aviation customers typically prioritize flexibility, balloon-propelled flights necessitate advance scheduling.  Despite these limitations, the potential market for affluent travel between cities like Dubai and Dallas is likely at least an order of magnitude larger than the satellite market.

Still, it would be disappointing if we can't make suborbital travel mainstream.   If we can launch our balloons without giant rockets, we can reduce cost.   After all, "the best part is no part," \cite{best_part_no_part} and a reusable rocket is one heck of a part.

\subsection{Lunar Volatiles for Externally Pulsed Propulsion}

\begin{figure}[htbp]
    \centering
    \includegraphics[width=0.5\linewidth]{images/Water Drawing From Moon.jpg}
    \caption{Lunar launched balloons replace Starship launched balloons \cite{earth_image} \cite{moon_image}}
    \label{fig:lunar_launched_balloons}
\end{figure}
Instead of using Starship, lunar volatiles extracted from the moon's permanently shadowed regions could fill our propulsive balloons. As shown in \autoref{fig:lunar_launched_balloons}, these balloons could be sent into a trans-lunar injection orbit that intersects our terrestrial suborbital rocket plane. Although balloon skins and electronics would likely still be sourced from Earth, volatiles constitute the majority of the balloons' mass. One approach to launching these balloons into orbit involves using rockets powered by lunar-derived propellants. The space community is enthusiastic about water electrolysis to produce and store cryogenic fuel for lunar rockets \cite{nasa_water}. However, this process is energy-intensive and requires complex cryogenic infrastructure, particularly for hydrogen fuel. Moreover, a reusable lunar rocket designed for repeated landings would require approximately \SI{6}{\km\per\second} of $\Delta v$, necessitating propellant mass fractions of about 75\% for hydrogen or 81\% for methane. While these figures are an improvement over Earth-launched rockets, the high propellant mass fractions remain a significant challenge.   

\subsection{Lunar Rockets without Lunar Rocket Fuel} \label{sec:lunar_rockets_no_fuel}
Instead of launching our lunar volatiles towards Earth with conventional rocket fuel, we launch off the moon using the same externally pulsed propulsion we've been discussing for terrestrial launches.   A rocket launched with externally pulsed propulsion from the lunar surface into a trans lunar injection orbit can burn its engines at periapsis around Earth.  Due to the Oberth effect, a small burn near Earth returns the rocket to the moon with a large velocity change.  As the rocket approaches the moon, it deploys its payload of balloons previously filled with lunar volatiles.  These returning balloons can push a new, more massive rocket filled with fresh volatiles off the moon.   We can then repeat this  cycle for exponential growth in lunar balloon launch mass capacity.   We've reduced combustible fuel requirements by burning our rockets near Earth.


\begin{table}[htpb!]
    \centering
    \begin{tabular}{|c|c|}
        \hline
        Before Collision & After Collision \\\hline
        \includegraphics[width=0.45\textwidth]{images/Pulsed Propulsion Chamber Before Impact.png} &
        \includegraphics[width=0.45\textwidth]{images/_Pulsed Propulsion Chamber After Impact.png} \\ \hline
         
    \end{tabular}
    \caption{A pulsed propulsion chamber.   Balloons enter, collide and produce thrust.   The valve, likely a fast rotating door synchronized with balloon arrivals, is optional. A small front opening will produce less thrust than a large rear opening.  If the explosion turns the gas into plasma, the valve could be magnetic rather than solid state.}
    \label{tab:pulsed_combustion_illustration}
\end{table}

However, we ideally don't want to make any chemical rocket fuel at all.   Let's again use external propulsion to launch our balloon deployment rocket from the lunar surface into a trans-lunar orbit.   We also launch a second set of balloons from the lunar surface into a retrograde trans-lunar orbit.   Using the millimeter precision navigation techniques we've discussed, we sequentially collide these retrograde balloons with more massive prograde balloons the main rocket is carrying.  As depicted in \autoref{tab:pulsed_combustion_illustration}, we trap the exploding gas from these collisions in a pulsed reaction chamber and redirect this gas for thrust, .   To leverage the Oberth effect and maximize gas collision velocity, we collide our balloons close to periapsis near Earth.

If our colliding balloons each travel at $v=\SI{11}{\kilo\meter\per\second}$, then from \autoref{eq:max_m_rb} derived in \autoref{sec:dv_effective},  the maximum theoretical combined effective exhaust velocity is $v_e = \frac{1}{2}v = \SI {5.5}{\kilo\meter\per\second}$.  Suppose this pulsed reaction has significant real world losses for an effective exhaust of $v_e = \SI{3}{\kilo\meter\per\second}$.   If the main rocket accelerates at it's Earth periapsis with a burn of \SI{1.5226}{\kilo\meter\per\second} it will reach the moon at \SI{6.42}{\kilo\meter\per\second}.   That's enough to start an exponential growth cycle where each loop around the moon launches about 1.6 times the starting mass.   If we do this once per month, we have increased our initial launch mass capacity by a factor of 1 million in only about 2.5 years. 

\subsection{Lunar Oxygen Has Mass And That's All It Needs}\label{sec:lunar_mining}
Lunar volatiles like water ice are ideal for filling balloons, but they are scarce.   However, oxygen is very common in lunar surface minerals.   Using solar energy and mining rovers derived from platforms like the IPex Pilot Excavator \cite{ipex_pilot_excavator}, we can produce oxygen.  On the volatiles depleted sunlit lunar surface, finding something to burn oxygen with for rocket propulsion is hard.   However, externally pulsed propulsion does not require oxygen combustion, just gas momentum.  In shadowed craters or artificial shade, lunar derived oxygen will liquify and perhaps even solidify for storage.  

We use some of our lunar balloons for the externally pulsed propulsion to lift terrestrial suborbital supply rockets into low Earth orbit.  These rockets then use conventional engines to gain \(\Delta v\) so they can resupply our lunar surface operations.   When they reach the moon, we also use externally pulsed propulsion to decelerate these supplies for lunar landing.  The deceleration balloons would previously have been launched from the moon's surface into an eccentric lunar orbit opposite to the incoming supplies' velocity.   We can optionally reduce terrestrial resupply requirements by embracing basic lunar manufacturing using metals like iron that are byproducts of lunar oxygen production.  These lunar metals could be shaped into pusher plate and pulsed reaction chamber components.  

We use the rest of our lunar balloons to exponentially grow our balloon mass capacity in orbit using the Oberth cycle discussed in \autoref{sec:lunar_rockets_no_fuel}.  Even though lunar regolith will likely degrade machinery somewhat faster than equipment would degrade on Earth, regolith resistant machines  may still enable propulsion at costs that make hypersonic travel scalable at competitive costs.

\subsection{Lunar Paleontology: Ancient DNA from Lunar Volatiles}\label{sec:jurassic_dark}
In \autoref{sec:lunar_rockets_no_fuel}, we explored using lunar volatiles from permanently shadowed craters for externally pulsed propulsion. These extremely cold craters could also indefinitely preserve DNA and other biomolecules from Earth impact ejecta \cite{dino_dna}. We could check the volatiles we're extracting to see if they contain homochiral organic molecules, which are only produced by living organisms.   Studying samples that contain homochiral molecules would offer a unique window into Earth's entire biological history, potentially revealing how life began. And, of course, the discovery of dinosaur DNA would surely delight \textit{Jurassic Park} \cite{jurassic_park} fans.

\section{Sorry, I Don't Need ISRU}
\subsection{The Need For Speed}\label{sec:no_isru_rocket}
If our propulsion balloons can reach high enough velocities, we could directly launch terrestrial payloads heavier than the balloons themselves.   This would eliminate the need for lunar ISRU. Repeatedly executing this launch cycle with ever larger (or more numerous) payloads would enable exponential growth in our launch capacity. Consider a rocket launched from Earth carrying pulsed propulsion payload.   We send $\frac{3}{4}$ our payload on a prograde trajectory with a solar periapsis similar to the Parker Space Probe and the remaining quarter on the opposing retrograde trajectory with the same Sun grazing periapsis.  These trajectories require high $\Delta v$ but we can use Jupiter gravity assists to achieve them (See \autoref{sec:jupiter_gravity_initial}).   These payloads release micro-thurst precision steered projectiles (see \autoref{sec:solid_balloons}) which then collide sequentially at periapsis in a pulsed propulsion chamber as described in \autoref{sec:dv_effective}. The resulting plasma can produce effective exhaust velocity of \SI{100}{\kilo\meter\per\second} or higher. Perhaps we accelerate from \SI{200}{\kilo\meter\per\second} to \SI{250}{\kilo\meter\per\second} with Earth crossing speeds around \SI{150}{\kilo\meter\per\second}.  At these Earth crossing velocities, incoming balloons could collide with balloons aboard a new terrestrial rocket in the rocket's propulsion chamber.  With the thrust from these collisions, our payload would then be sent on prograde and retrograde orbits with low solar periapsis, repeating the cycle.  If desired, we can optimize exhaust velocity further by lowering the periapsis in steps, allowing our payload to fall significantly towards the Sun after each step.  This way, collisions after each periapsis reduction occur at progressively higher velocities for greater effective exhaust velocity.   If we double payload mass every 3 months, we increase our initial payload by a factor of 1 million in about 6 years. 

\subsection{Radiative Differences from Project Orion}
Project Orion required shaped explosives to redirect plasma towards the pusher plate because 80\% of the bombs' energy became black body X-ray thermal radiation \cite{toughsf_cassaba_howitzer}.  In contrast, our proposed system anticipates minimal radiative losses from collisions, even at speeds matching the Parker Solar Probe. Though detailed computer analysis is needed for full confirmation, our reasoning for this low loss stems from
\begin{itemize}
    \item The much lower atomic mass of the colliding atoms compared to fissile atoms, leading to proportionally lower temperatures at equivalent velocities.
    \item The mean velocity of colliding atoms is likely about $\frac{1}{5}$ that of atoms in an inefficient 1\% yield bomb, generating peak temperatures at least 25 times lower
    \item Combined, lower mean velocity and atomic mass imply temperatures might be 200 times less in our collision than in a bomb.  Radiative power rises with the fourth power of temperature ($T^4$), so peak radiation flux is likely around 1 billion times less than the least efficient nuclear bombs.
    \item The colliding masses should be much smaller than a nuclear bomb, with a lower radiative surface area.   As the collision plasma expands adiabatically, its volume increases proportionally faster than a larger bomb plasma would at the same linear expansion velocity.  Temperature would decrease faster due to this faster proportional change in volume. 
\end{itemize}

Lower temperatures generate longer-wavelength blackbody X-rays compared to those from bombs. This means materials with lower atomic numbers (lower Z) can absorb this radiation and keep the plasma opaque. While Project Orion favored expensive and difficult to work with tungsten \cite{toughsf_cassaba_howitzer}, these longer wavelengths should allow for the use of cheaper elements to achieve opaque plasma.

\subsection{Solids Rather than Low Density Balloons}\label{sec:solid_balloons}
Given the extreme velocities at periapsis, we no longer collide low density balloons because solid objects vaporize into opaque plasma.   The real-world efficiency of these plasma explosions is likely high because the extreme temperatures should bring the entire plasma to thermal equilibrium before significant expansion occurs.

We'd make these solid "balloons" primarily from cheap low atomic number (low Z) materials like boron nitride, motor oil, and graphite.  A small amount of cheap higher Z materials like iron would be included to ensure opacity to the blackbody X-rays produced.  
All the balloon's mass contributes to propulsion, irrespective of its composition.  The balloons can contain electronics, heat shielding, and some liquid that passively or actively cools the skin through convection.     They can also partially be made from batteries so that solar power is not needed too close to the Sun.    

\subsection{Navigation Challenges Near Periapsis}\label{sec:periapsis_challenges}
Close solar periapsis involves some unique challenges.   Obviously, any sensor dependent on visible light will need to filter interfering solar illumination.   Instruments will need to handle heat and solar wind.  Relativisitic influences preturb the craft more than they would on Earth.  High collision speeds demand extremely accurate navigation.   The pulsed propulsion chamber may ablate slightly on each pulse.  The impacting projectiles must be miniaturized to at most a few kilograms to contain the immense energy from explosions at these velocities. 

Perhaps the most significant nuance is the need to adjust for solar weather.  Since this weather influences luminosity, it also influences radiation pressure.   Fortunately, neural networks appear capable of improving weather predictions \cite{lam2023learning}.   

Spacing balloons sequentially far apart  may be hard.  Objects far apart at periapsis get closer together as radial distance grows.  As a result, Earth distance propulsion balloons would need to impart relatively high acceleration in short times.  However, the terrestrial payloads we're directly accelerating this way should be unmanned.   

When we lift high value payloads like astronauts into Low Earth Orbit, we use two sets of propulsive balloons. First, our Earth crossing balloons from the Sun push unmanned payloads into eccentric orbits around Earth. Then these eccentric payloads deploy balloons to push vital cargo like people or  satellites just as we discussed in \autoref{sec:200_mile_high}.

\subsection{Initial Periapsis Orbits with Jupiter Gravity Assists} \label{sec:jupiter_gravity_initial}
A final issue is placing the initial payload in a retrograde low periapsis solar orbit.  The best approach for this is likely using a gravity assist from Jupiter, which was the original plan for the Parker Space Probe mission \cite{mccomas2005solar}. Note that our payload can include large batteries and radiation shielding.  The batteries obviate the need for solar power far from the Sun, and the radiation shielding allows our gravity assist to occur at low Jupiter radial distance.   

Optionally we can push more mass into the initial orbit by using Jupiter gravity assist (possibly with a burn at Jupiter to leverage it's Oberth effect) to place propulsion balloons into a retrograde Hohmann trajectory back to Earth.   They would cross Earth at around \SI{69}{\kilo\meter\per\second}, which could push a larger terrestrial payload onto  prograde and retrograde colliding trajectories at low solar periapsis.

\section{One Hiroshima Per Second. The World Set Free}\label{sec:world_set_free}
In \autoref{sec:no_isru_rocket}, we propelled balloons to \SI{150}{\kilo\meter\per\second} at \SI{1}{\astronomicalunit} using solar gravity and externally pulsed propulsion. A \SI{60}{\tera\watt} pulsed heat source, which is equivalent to one Hiroshima bomb per second \cite{hiroshima}, requires harnessing the kinetic energy of just \SI{5.3}{\tonne\per\second} of propulsion mass at these speeds. For context, humanity's total power consumption is approximately \SI{20}{\tera\watt} \cite{owid-energy-production-consumption}. Given that typical power plants are roughly one-third efficient, a single one of these pulsed heat sources could meet the energy needs of all human civilization.  Remarkably, this concept evokes H.G. Wells's prophetic vision in his 1913 novel, \textit{The World Set Free}, where he imagined atomic bombs as a "blazing continual explosion" \cite{wells1914world}.  

Using kinetic projectiles to generate terrestrial power only makes sense if it achieves massive economies of scale that outperform the linear costs and land use efficiency of conventional renewables.  Let's explore how to build such a truly ambitious megastructure, and then explain why it's economically attractive in \autoref{sec:strawway_economics}.     

\subsection{Straw Ways To Heaven}\label{sec:straw_way_to_heaven}
The most common proposal for utilizing space-based power on Earth involves wireless power transfer from orbiting power plants to terrestrial rectenna arrays.  Unfortunately, long distance wireless transfer requires breakthroughs in wireless beam focusing \cite{space_beaming_power} and large radiators for dissipating heat. Moreover, the required wireless transmission infrastructure, both in space and on Earth, would be colossal.  Given these challenges, let's investigate terrestrial solutions which directly convert kinetic impact energy to electricity on Earth.

To deliver kinetic energy projectiles to Earth, the projectiles must be able to withstand atmospheric entry.  One solution is to build a vacuum tunnel, extending from Earth's surface to space, precisely aligned to allow incoming projectiles to pass through unimpeded.  The tunnel is suspended from an anchor point as depicted in \autoref{fig:straw_way_to_heaven}.   Enthusiasts have proposed numerous physically plausible structures that could support an anchor hoisted above Earth's atmosphere. These include Lofstrom Launch Loops \cite {lofstrom_loop}, tethered and global orbital rings, space elevators, and space fountains \cite{isaac_arthur_megastructure_complation}.  However, these ambitious ideas generally suffer from very low Technology Readiness Levels (TRLs), meaning they are far from practical implementation.

\begin{figure}[htpb]
    \centering
    \includegraphics[width=0.5\linewidth]{images/Straw Way To Heaven.png}
    \caption{Vacuum Tube Suspended From Cables Allows Projectiles Through the Atmosphere}
    \label{fig:straw_way_to_heaven}
\end{figure}


Fortunately, externally pulsed propulsion offers a promising path to position an anchor point above Earth. While this technology also lacks significant technical maturity, we would have already developed it to highly reliable standards if we were using it to direct projectiles at Earth using the techniques from \autoref{sec:no_isru_rocket}}.  

A heavy lift rocket, such as SpaceX's Super Heavy \cite{spacex_super_heavy}, lifts the anchor point (and the vacuum tube it is connected to) to its initial height.  We then indefinitely hover the anchor point using equal mass balloons that collide head on in a pulsed reaction chamber (described in \autoref{sec:lunar_rockets_no_fuel}), redirecting thrust downward.  A reaction control system on the anchor point adjusts the pulsed reaction chamber so it stays precisely positioned for the next balloon collision.  If these balloons meet at $v=\SI{11}{\kilo\meter\per\second}$ at an average mass flow rate of \SI{1}{\tonne\per\second} they can theoretically suspend more than \SI{1100}{\tonne}.   

In reality, energy losses would reduce thrust.   For robust operations, we would need to lift redundant anchor points and vacuum tubes.  Together, redundancy and energy losses imply we will need additional external propulsion collision mass per second.  Nevertheless, lifting a few tons per second of propulsion balloon mass from Earth (as in \autoref{sec:200_mile_high}) or the moon (as in \autoref{sec:lunar_rockets_no_fuel}) seems economically viable if we use our vacuum tube to cleanly produce all the energy required by mankind. 

\subsection{Vacuum Tube Details}
Our proposed vacuum tube should have tapered walls since preserving a vacuum at higher elevations only requires simple thin barriers. The tube's optimal form will be maintained by cables with adjustable lengths connected to a central anchor point. The tube can curve to compensate for Coriolis forces during projectile flight. The anchor point itself will rotate throughout the day to track the Sun.

To keep the vacuum tube aloft, external propulsion will be employed, even during inactive nighttime hours. Alternatively, for maintenance, the tube could be lowered at night and then re-hoisted by a heavy-lift rocket the following day.

Creating a high vacuum in long tunnels presents a challenge, though it's successfully achieved in facilities like the Large Hadron Collider. Fortunately, our system benefits from the projectiles themselves acting as a natural vacuum pump. Stray atoms will tend to be bounced downwards by  projectile collisions. While some atoms may sputter off the projectiles, these will also be directed downwards. Ultra high vacuum equipment at the bottom of the tunnel will effectively remove these atoms, ensuring the vacuum remains strong.

Projectiles that catastrophically miss the vacuum tube cause only moderate Chelyabinsk level \cite{popova2021chelyabinsk} ground effects because their extreme velocity guarantees they burn up at high altitudes.   Assuming our vacuum tubes are in remote areas, this causes minimal disturbances to population centers.   The vacuum tube itself is likely destroyed, but redundant vacuum tubes would then be moved into place.

\subsection{Power Plant and Vacuum Air Lock Exchange Details}
Beneath the vacuum tube, we propose excavating a massive steam chamber designed to vaporize incoming projectiles. These chambers would be substantial construction projects, engineered to withstand sudden changes in pressure and temperature.  Suppose a \SI{0.1}{\cubic\kilo\meter} steam chamber containing \SI{50}{\kilo\gram\per\cubic\meter} steam receives a \SI{600}{\tera\joule} kinetic impact.   As a simplification, let's say the steam has a constant specific heat of \SI{3.5}{\kilo\joule\per(\kilo\gram\celsius)}.  In this case the temperature change is 
$\frac{\SI{600}{\tera\joule}}{\SI{5e9}{\kilo\gram}\times\SI{3500}{\joule\per(\kilo\gram\kelvin)}} = \SI{34.2}{\celsius}$.  An impact of this magnitude every 10 seconds produces the heat to continuously provide all humanity's energy.  This steam chamber would be connected to thousands of turbines and a massive cooling loop to harness and dissipate this energy.  

Releasing large amounts of heat into the atmosphere would lead to unacceptable local temperature increases. Using a coolant loop that exchanges heat with deep ocean water will also have environmental consequences. Spreading $\SI{40}{\tera\watt}$ of heat over $\SI{10}{\cubic\kilo\meter}$ of deep ocean water would increase its temperature by $\frac{\SI{40}{\tera\watt}\cdot\SI{3600}{\second\per\hour}}{\SI{1e13}{\kilo\gram}\cdot\SI{4186}{\joule\per\kilo\gram\per\celsius}} = \SI{3.44}{\celsius\per\hour}$.  This rapid temperature change could severely damage local marine habitats.

However, considering the limited marine life found far above the ocean floor and below the sunlit zone, the broader deep sea ecology beyond the directly heated zone would remain stable. When compared to the widespread impacts of global anthropogenic climate change or even the significant displacement of wildlife from building solar farms, the environmental harm from this specific heat rejection method appears minimal.


\begin{figure}[htpb]
    \centering
    \includegraphics[width=0.5\linewidth]{images/Power Plant Cooling and Generators.png}
    \caption{The Power Plant and Vacuum Air Lock Exchange.  Air Lock segments are rotated to give time to restore high vacuum between uses}
    \label{fig:power_plant_vacuum_exchange}
\end{figure}

An air lock separates the steam chamber from the rest of the tunnel as shows in \autoref{fig:power_plant_vacuum_exchange}.   A spinning door with a tip speed of \SI{300}{\meter\per\second} could close a \SI{30}{\centi\meter} diameter aperture in \SI{1}{\milli\second}.  Even a \SI{5}{\meter} long air lock is long enough to prevent steam molecules from traversing the air lock in this short time.   We swap in a new air lock segment between kinetic projectile impacts.  This gives us time to restore the high vacuum in the airlock segment before it's used again.   


\subsection{With Economies of Scale, Our Ambitions Can Scale Too}\label{sec:strawway_economics}

While a global power plant requires a massive initial capital investment, its costs scale favorably with output. The cost of suspending a large vacuum tube depends very little on the plant's generating capacity. The cost of deep sea cooling loops would also be amortized across the entire planetary scale facility.  A multi-terawatt installation could achieve significant per-watt cost reductions through bulk procurement of thousands of identical turbine and generator units. 

Mass production significantly reduces the cost of projectile electronics and micro thrusters. The cost of these components per projectile also remains nearly the same as individual projectile size increases.  Ten seconds of planetary scale power corresponds to $\SI{20}{\tera\watt} \times \SI{10}{\second} \approx \SI{5.56e7}{\kilo\watt\hour} = \$555,556$ at 1 cent per kilowatt-hour.  This energy cost would likely dwarf the material costs of the projectiles themselves.      

Power from our plant could be globally distributed by super tanker sized batteries that connect to urban power grids at coastal ports.  While the cost of utility-scale batteries is progressively declining, it remains a significant hurdle. Current battery designs prioritize efficient energy storage and release. However, if energy efficiency becomes less critical, we could design cheaper low performance batteries. Given that our power source exhibits strong cost efficiencies, we could simply increase the power output to compensate for any energy losses incurred by using these more affordable batteries for distribution. 

Batteries are environmentally attractive, but our world still depends heavily on fossil fuels.  Very low cost power can cheaply produce hydrocarbon fuels from water and carbon dioxide.  The hydrogen can come from electrolysis, while carbon dioxide comes from biomass, heating carbonates, industrial waste, or direct air/ocean capture.

Though the power plant and distribution infrastructure would cost billions to construct and operate, the revenue generated is staggering. Ultimately, we'd spend billions to make trillions annually from a planetary-scale plant.

\subsection{To Solve Climate Change, Make The Earth A Death Star}\label{sec:death_star}
A Death Star \cite{death_star} is a giant sphere with a  narrow 1 meter opening you can shoot something down to create an explosion inside the main reactor.   Building a Death Star from scratch  would be a huge undertaking.   Fortunately, if we build the Straw Way To Heaven described in \autoref{sec:straw_way_to_heaven}, we're living on one.  Although our Straw Way generator produces clean energy from incoming projectiles, it's power production is restricted to daytime hours when the straw can face the sun.   Also, a limitless clean energy source would only slow climate change.   Industrial carbon emissions from steel, cement, agriculture, and other processes would still heat the planet \cite{steel_and_cement}.   

\subsubsection{That's No Moon \cite{starwars1977}.  Oh Wait, It's Literally Our Moon}
Projectiles made from lunar materials could enter a terrestrial Straw Way from high angles of attack around the clock.  Lunar regolith is primarily carbon depleted rock \cite{mckay1991lunar}.  If this rock was heated, vaporized, and pressurized as it fell into deep ocean water, it would rapidly absorb carbon dioxide and  negate industrial carbon pollution.

To generate \SI{60}{\tera\watt} of heat from lunar bombardment, we'd need an astonishing \SI{1}{\kilo\tonne\per\second} of lunar mass to pass through the terrestrial vacuum tube. Such a vast quantity is daunting, but the momentum for launching off the moon could come from kinetic projectiles inbound from the Sun (as described in \autoref{sec:no_isru_rocket}). Beyond providing thrust, these projectiles aimed at the Moon could vaporize lunar regolith. The resulting thermal energy could be harnessed for electricity generation or to power chemical and industrial processes directly. This lunar power source could support large scale mining and production of the oxygen gas used in externally pulsed propulsion  (as discussed in \autoref{sec:lunar_mining}).

\subsection{Fusion Propulsion and Epstein Drives}\label{sec:epstein_drives}
The fast projectiles from \autoref{sec:no_isru_rocket} could propel a spacecraft around the inner solar system at relatively high speeds.  Ideally though, we'd want torch ships that travel under constant Earth gravity acceleration between the planets as envisioned in \textit{The Expanse} \cite{Corey2012Drive}.  Advanced heat shields like solar white should enable spacecraft to approach the Sun to 2 solar radii and potential head on collision velocities exceeding \SI{800}{\kilo\meter\per\second} in the target rocket's reference frame.  

\begin{figure}[htpb]
    \centering
    \includegraphics[width=0.5\linewidth]{images/Fusion Impactors.png}
    \caption{A lead tamper generates immense shock heating around fusion fuel and compresses it to ignition conditions during head-on collisions}
    \label{fig:fusion_tamper}
\end{figure}

A 1979 Los Alamos Impact Fusion workshop concluded speeds of just \SI{200}{\kilo\meter\per\second} might be sufficient for precompressed gram sized projectiles \cite{impactfusion1979}.   Larger projectiles would likely be easier to confine long enough for ignition.   A lead tamper, as depicted in \autoref{fig:fusion_tamper} surrounding a few hundred grams of fusion fuel could ignite fusion explosions and trigger immense exhaust velocities.  
This could propel projectiles towards spacecraft at velocities of up to \SI{20,000}{\kilo\meter\per\second}, especially if the fusion fuel is mostly aneutronic.   A seed of deuterium-tritium could ignite a larger deuterium and helium-3 mixture to maximize thrust.   As the spacecraft accelerate, they can also reach speeds where they trigger fusion by crashing into stationary fusion fuel targets at speeds exceeding \SI{1000}{\kilo\meter\per\second}.

\section{When We Get Greedy, We'll Go To Phoebe}\label{sec:greedy_phoebe}
In \autoref{sec:no_isru_rocket} we discussed using Sun grazing Oberth maneuvers to accelerate projectiles to \SI{150}{\kilo\meter\per\second} at \SI{1}{\astronomicalunit}. These high velocities give the projectiles enough energy for the high $\Delta v$ required to place the next suborbital rocket into  prograde and retrograde orbits near the Sun. Such close approaches introduce significant challenges, as detailed in \autoref{sec:periapsis_challenges}. These difficulties could increase projectile costs compared to those making less extreme solar approaches.

Orbits with higher apoapsis require less $\Delta v$ to shed orbital angular momentum and significantly lower their periapsis. For example, transferring to a solar-impact trajectory from Saturn requires only about \SI{10}{\kilo\meter\per\second}, compared to roughly \SI{30}{\kilo\meter\per\second} from Earth. This makes Saturn a more efficient departure point for reaching the inner solar system or the Sun.

Saturn’s irregular moon Phoebe lies just $\SI{180}{\degree} - \SI{173.4}{\degree} = \SI{6.6}{\degree}$ off the solar ecliptic, making it well aligned for interplanetary transfers. It orbits at high elevation with a relatively slow velocity relative to Saturn, offering low $\Delta v$ transfers to orbits reaching close to Saturn.

A prograde rocket launched from Phoebe could collide with retrograde payloads also launched from Phoebe. These collisions could occur in a pulsed propulsion chamber near Saturn, enabling high efficiency Oberth maneuvers. With precise timing to avoid Saturn’s rings, this method could launch spacecraft toward the inner solar system with significantly enhanced velocity.

These rockets can also return to Phoebe at high speeds to launch more mass with pulsed propulsion, enabling a bootstrapped launch mass growth cycle similar to the terrestrial moon cycle discussed in \autoref{sec:lunar_rockets_no_fuel}. Power for mining Phoebe come from solar energy concentrated from lenses built with Phoebe’s native ice.  Alternatively, thermal energy generated by returning impactors could power mining on Phoebe.

While Phoebe’s low gravity may pose challenges for mining and anchoring infrastructure, the more massive Iapetus is also a good option. Similarly, Jupiter’s moon Himalia may support analogous maneuvers if exploration discovers  it has significant volatiles.    
\subsection{Mining Helium-3}
Helium-3 for aneutronic fusion may be available on the Earth's moon \cite{esa_helium3_mining}, but likely not in quantities to support large scale fast interplanetary transport like discussed in \autoref{sec:epstein_drives}.   The gas giants likely contain inexhaustible supplies of helium-3 \cite{palaszewski2005atmospheric}.   A nuclear thermal rocket or scram jet could extract helium 3 and hydrogen while flying in Saturns atmosphere and then jump a few thousand kilometers above the atmosphere.  It could then separate from a payload carrying Helium3, falling back to mine more helium3.  An accident on such a rocket would cause no harm, because waste would simply fall into Saturn's core. Propulsion balloons from Phoebe could then send Helium 3 into the inner solar system for use in fusion rockets.


\section{War, Policy, and Pulsed Propulsion}
\subsection{Fusion Rockets and the Outerspace Treaty}
The Outerspace Treaty \cite{outer_space_treaty} prevents nuclear explosions in space.  The legality of pure fusion rockets discussed in \autoref{sec:epstein_drives} may need clarification.

\subsection{Hackers Must Be Prevented from Causing Catastrophic Explosions}
As previously discussed, the energy projectiles for the \textit{Death Star} discussed in \autoref{sec:death_star} are intentionally designed to burn up high in the atmosphere without causing damage in case of a mishap.  However, there is a danger a hacker could manipulate microthrusters on the projectiles to enter the atmosphere simultaneously rather than in sequence, leading to larger Tunguska \cite{longo2007tunguska} sized explosions.  We can mitigate this risk with standard practices like using memory safe languages and authenticated communication.

\subsection{Risk of Balloon Sabotage}
External pulsed propulsion might be useful for launching military assets like Golden Dome \cite{lockheed_martin_golden_dome} or the Brilliant Boulders \cite{brilliant_boulders} discussed on the blog related to this paper.   The balloons described in \autoref{sec:lunar_rockets_no_fuel} are likely vulnerable to rupture from high powered terrestrial lasers by a military adversary.  Potentially a thin reflective shield should be attached to the balloon and discarded along with other solid componenets right before impact.   Eventually, militaries will need to defend their payloads by destroying sabotage weapons however.


\appendix 
\section{Derivation of Balloon Mass/Rocket Mass Continous Approximation}\label{sec:balloon_ratio_approximation}  We want to approximate the ratio of the total balloon mass \(m_b\) all traveling at velocity \(v_b\) for a sequence of balloons to push a rocket with mass \(m_r\) with initial velocity \(v_{ri}\) to final velocity \(v_{rf}\).   Let's initially naively assume every collision is perfectly elastic in one dimension.   We'll use calculus to get a  closed form expression by solving for a rocket  continously bombarded by infinitestimal balloons each with mass \(dm_b\).   (Note: Grok \cite{grok}  helped with some of the math for this derivation and parts of this derivation are copied from Grok  results directly)

The initial velocities are
\begin{itemize}
\item Mass \( m_r \): Velocity \( v_r \).
\item Balloon: Mass \( dm_b \), velocity \( v_b \).
\end{itemize}
After the collision, let:
\begin{itemize}
    \item Mass \( m_r \): Velocity \( v_r + dv_r \).
    \item Balloon: Velocity \( v_b' \).
\end{itemize}
\subsection{By conservation of momentum} \[
m_r v_r + dm_b v_b = m_r (v_r + dv_r) + dm_b v_b'
\]
\[
m_r v_r + dm_b v_b = m_r v_r + m_r dv_r + dm_b v_b'
\]
\[
dm_b v_b = m_r dv_r + dm_b v_b'
\]
\begin{equation}
m_r dv_r = dm_b (v_b - v_b') \label{eq:momentum}
\end{equation}

\subsection[Velocity change of rocket mass]{Velocity change of \(m_r\)}
For an elastic collision
\begin{equation}
    v_b'\ = \frac{2m_r}{m_r+m_b}v_r + \frac{m_b-m_r}{m_r+m_b}v_b \label{eq:vb_prime_full_momentum}
\end{equation}
Since \(m_r \gg m_b\)   we plug \(m_b = 0\)  into \autoref{eq:vb_prime_full_momentum} and find 
\begin{equation}
v_b' = 2v_r - v_b  \label{eq:vb_prime_result}
\end{equation}      

Substituting \(v_b'\) from  \autoref{eq:vb_prime_result} into \autoref{eq:momentum}:
\[
m_r dv_r = dm_b (v_b - (2 v_r - v_b))
\]
\[
m_r dv_r = dm_b (v_b - 2 v_r + v_b)
\]
\[
m_r dv_r = dm_b (2 v_b - 2 v_r)
\]
\[
m_r dv_r = 2 dm_b (v_b - v_r)
\]
\begin{equation}
dv_r = \frac{2 (v_b - v_r)}{m_r} dm_b \label{eq:vchange_mr}
\end{equation}

\subsection{Integrate over many collisions}
The velocity of \( m_r \) changes from \( v_{ri} \) to \( v_{rf} \) as more balloons collide. Integrate \autoref{eq:vchange_mr}:
\[
\int_{v_{ri}}^{v_{rf}} dv_r = \int_0^{m_b} \frac{2 (v_b - v_r)}{m_r} dm_b
\]
The left-hand side is:
\[
\int_{v_{ri}}^{v_{rf}} dv_r = v_{rf} - v_{ri}
\]
For the right-hand side, treat \( v_r \) as a function of the accumulated balloon mass \( m_b \). However, we need to express \( v_r \) in terms of \( m_b \). From \autoref{eq:vchange_mr}:
\begin{equation}
\frac{dv_r}{v_b - v_r} = \frac{2 dm_b}{m_r}\label{eq:dvr_velocity_relation}
\end{equation}

Since \(v_b\) is constant,  \(dv_b = 0\) and  \(dv_r = -d(v_b-v_r)\), so we can rewrite \autoref{eq:dvr_velocity_relation} as:
\[
-\frac{d(v_b - v_r)}{v_b - v_r} = \frac{2 dm_b}{m_r}
\]
Integrate both sides:
- Left: \( \int_{v_b - v_{ri}}^{v_b - v_{rf}} -\frac{d(v_b - v_r)}{v_b - v_r} = \int_{v_{ri}}^{v_{rf}} \frac{dv_r}{v_b - v_r} \)
\[
= \left[ -\ln|v_b - v_r| \right]_{v_{ri}}^{v_{rf}} = \ln \left| \frac{v_b - v_{ri}}{v_b - v_{rf}} \right|
\]
- Right: \( \int_0^{m_b} \frac{2 dm_b}{m_r} = \frac{2 m_b}{m_r} \)
Equate:
\[
\ln \left| \frac{v_b - v_{ri}}{v_b - v_{rf}} \right| = \frac{2 m_b}{m_r}
\]
Solve for \( m_b \):
\[
\left| \frac{v_b - v_{ri}}{v_b - v_{rf}} \right| = e^{2 m_b / m_r}
\]
\begin{equation}
m_b = \frac{m_r}{2} \ln \left| \frac{v_b - v_{ri}}{v_b - v_{rf}} \right| 
\end{equation}
\subsection{Interpret the Absolute Value}   
The absolute value accounts for the direction of velocities:
- If \( v_b > v_{ri} \) and \( v_b > v_{rf} \),  then \( v_b - v_{ri} \) and \( v_b - v_{rf} \) are positive, so:
\[
m_b = \frac{m_r}{2} \ln \left( \frac{v_b - v_{ri}}{v_b - v_{rf}} \right)
\]
\subsection{Compute Balloon to Rocket Mass Ratio, with a fudge factor for real world losses}

We can then solve for the total balloon to rocket ratio, with a fudge factor e between 0 and 1 to account for the coefficient of restitution and imperfect spread of gaseous volatiles off the pusher plate.   Let's naively assume this fudge factor is the same for each collision even though in practice it would likely vary as the relative velocity of the balloons and the rocket change.
\begin{equation}
\frac{m_r}{m_b} = \frac{2e}{ln(\frac{v_b-v_{ri}}{v_b-v_{rf}})}\label{eq:balloon_ratio}
\end{equation}

\section{Deriving Effective Exhaust Velocity $v_e$ for Idealized 100\% Efficient Prograde/Retrograde Balloon Collision Rocket Thrust}\label{sec:dv_effective}
Suppose a retrograde balloon with $m_{rb}$ and a prograde balloon with $m_{pb}$ collide in a pulsed rocket propulsion chamber, as depicted in \autoref{tab:pulsed_combustion_illustration}.  We want to find the maximum exhaust velocity $v_e$ for the prograde rocket  when engine efficiency is 100\% and we expel all the combined gas behind us (in the retrograde direction).   For simplicity we say 
\begin{equation}
m_{rb} + m_{pb} = 1\label{eq:mass_is_1}
\end{equation} 
Each balloon has velocity $v$ so the retrograde balloon has velocity $2v$ in the reference frame of the prograde balloon.   In the prograde reference frame, kinetic energy is
\begin{equation}
E = \frac{m_rb (2v)^2}{2} = 2m_{rb}v^2\label{eq:ke_balloons}
\end{equation}

Let $v_g$ be gas exhaust velocity expelled from the rocket, and assume this mass is infinitesimal compared to the entire prograde payload.  Applying mass from \autoref{eq:mass_is_1}
 and kinetic energy from \autoref{eq:ke_balloons}, we have 
 \begin{equation}
 2m_{rb}v^2= \frac{v_g^2}{2}
 \end{equation}
 and solving for $v_g$ we get 
 \begin{equation}
 v_g = \sqrt{4v^2m_rb} = 2v\sqrt{m_{rb}} \label{eq:vg_result}
 \end{equation}
 Total momentum change is 
 \[(m_{rb} + m_{pb})v_g - 2vm_{rb} = v_g-2vm_{rb} = 2v(\sqrt{m_{rb}} - m_{rb}) \]
 Using calculus, its straightforward to show we maximize exhaust velocity when 
 \begin{equation}
 m_{rb} = \frac{1}{4}, \label{eq:max_m_rb}
 v_g = v,
 v_e= \frac{v}{2}
 \end{equation}
 
%Bibliography
\bibliographystyle{unsrt}  
\bibliography{references}  


\end{document}
