\documentclass{article}


\usepackage{PRIMEarxiv}

\usepackage[utf8]{inputenc} % allow utf-8 input
\usepackage[T1]{fontenc}    % use 8-bit T1 fonts
\usepackage{hyperref}       % hyperlinks
\usepackage{url}            % simple URL typesetting
\usepackage{booktabs}       % professional-quality tables
\usepackage{amsfonts}       % blackboard math symbols
\usepackage{nicefrac}       % compact symbols for 1/2, etc.
\usepackage{microtype}      % microtypography
\usepackage{siunitx}
\usepackage{hyperref}
\sisetup{per-mode=symbol}
\usepackage{lipsum}
\usepackage{parskip}
\usepackage{fancyhdr}       % header
\usepackage{graphicx}       % graphics
\graphicspath{{images/}}     % organize your images and other figures under media/ folder

%Header
\pagestyle{fancy}
\thispagestyle{empty}
\rhead{ \textit{ }} 

% Update your Headers here
\fancyhead[LO]{Aim Is All You Need}
% \fancyhead[RE]{Firstauthor and Secondauthor} % Firstauthor et al. if more than 2 - must use \documentclass[twoside]{article}



  
%% Title
\title{Aim Is All You Need}

\author{
  Seth Katz \\
  \texttt{\{katzseth22202@gmail.com} \\
}


\begin{document}
\maketitle

\begin{abstract}\label{sec:abstract}
 In 2017, Google Research published ``Attention is All You Need" \cite{vaswani2023attentionneed}.  This paper introduced the Transformer, which let neural networks capture long range dependencies between words in a sentence or pixels in an image.   Just a few years later, OpenAI developed tools like ChatGPT that resemble hypothetical early prototypes of the computers in Star Trek.

But here's the bittersweet truth:  While our screens flicker with progress, the tangible realms of space, energy and paleontology remain comparatively stagnant.

\textbf{Enter our mission}

This paper's goal is to enable the progress on physics to catch up to our IT progress.  Our journey requires applying a single unifying idea that is much simpler than attention - aim.   Specifically, we'll argue that if we can master the technology to precisely throw collections of fast objects at a specific target in space, that's all we need to achieve some truly groundbreaking geek milestones.   Just as neural networks enabled consumer AI, they potentially allow us to achieve the precision control to make this external pulsed propulsion concept feasible.

\textbf{A Checklist of Grand Challenges Aim Can Solve}
\begin{itemize}
    \item 
Suborbital Transit:  We'll create a viable suborbital travel market allowing passengers to reach anywhere in the world in less than 2 hours
 \item Rocket Revolution: We'll end the tyranny of the rocket equation.   We'll still use small rockets, but giant rockets with high propellant mass fraction will no longer be needed to reach orbit
  \item Jurassic Dark: We'll discover concrete evidence for how life originated on earth and a genetic record of extinct species from ancient geological periods, like the dinosaurs. \item We'll also create lunar paleontology, a new science field uncovering Earth's biological history in the permanently shadowed regions of the moon
  \item Carbon Cancelled: We will solve our energy problems with carbon negative fuel that absorbs the carbon dioxide produced by industry, all while using minimal land and resources
  \item Moon mining:  We'll develop off world ISRU technology first on our moon and then on icy moons, ideally Saturn's move Phoebe
\end{itemize}
Note, this paper summarizes some of the ideas in the blog ``Aim Is All You Need"\cite{aim2024}.
\end{abstract}

\section{Introduction}
Project Orion, conceived in the 1950s, remains the sole known propulsion technology offering both high specific impulse and high thrust \cite{projorion}. Its mechanism involved propelling a spacecraft by directing hypervelocity plasma from nuclear explosions onto a pusher plate. Despite its theoretical promise, Orion faced insurmountable challenges related to political feasibility, radioactive fallout, and the impractical mass requirements stemming from the high minimum yield of nuclear explosions. Nevertheless, Orion unequivocally validated the physics of hypervelocity pulsed propulsion.

Our approach replaces nuclear bombs with precisely aimed, hypervelocity gas balloons impacting a rocket's pusher plate. These balloons can be scaled to practical sizes, and unlike nuclear explosions, small hypervelocity impacts could potentially be efficiently contained within a pulsed reaction chamber. This method would circumvent the restrictions of the Outer Space Treaty \cite{outer_space_treaty} and eliminate radioactive byproducts.

Rather than detailing a specific algorithm, this paper broadly argues for the pivotal role of neural networks in enabling the precision control necessary for this hypervelocity pulsed propulsion concept. Subsequently, we explore the radical applications enabled by this technology, as outlined in the abstract\ref{sec:abstract}.



%Bibliography
\bibliographystyle{unsrt}  
\bibliography{references}  


\end{document}
