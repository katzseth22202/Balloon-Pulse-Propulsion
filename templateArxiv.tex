\documentclass{article}


\usepackage{PRIMEarxiv}

\usepackage[utf8]{inputenc} % allow utf-8 input
\usepackage[T1]{fontenc}    % use 8-bit T1 fonts
\usepackage{hyperref}       % hyperlinks
\usepackage{url}            % simple URL typesetting
\usepackage{booktabs}       % professional-quality tables
\usepackage{amsfonts}       % blackboard math symbols
\usepackage{nicefrac}       % compact symbols for 1/2, etc.
\usepackage{microtype}      % microtypography
\usepackage{siunitx}
\usepackage{hyperref}
\sisetup{per-mode=symbol}
\usepackage{lipsum}
\usepackage{parskip}
\usepackage{fancyhdr}       % header
\usepackage{graphicx}       % graphics
\graphicspath{{images/}}     % organize your images and other figures under media/ folder

%Header
\pagestyle{fancy}
\thispagestyle{empty}
\rhead{ \textit{ }} 

% Update your Headers here
\fancyhead[LO]{Aim Is All You Need}
% \fancyhead[RE]{Firstauthor and Secondauthor} % Firstauthor et al. if more than 2 - must use \documentclass[twoside]{article}



  
%% Title
\title{Aim Is All You Need}

\author{
  Seth Katz \\
  \texttt{\{katzseth22202@gmail.com} \\
}


\begin{document}
\maketitle

\begin{abstract}\label{sec:abstract}
 In 2017, Google Research published ``Attention is All You Need" \cite{vaswani2023attentionneed}.  Their paper introduced the Transformer, which let neural networks capture long range dependencies.   Just a few years later, OpenAI developed tools like ChatGPT \cite{chatgpt} that resemble hypothetical early prototypes of the computers in Star Trek \cite{startrek}.

But here's the bittersweet truth:  While our screens flicker with progress, the tangible realms of space, energy and paleontology remain comparatively stagnant.

\textbf{Dude, Where's my spaceeship?}

This paper's goal is to enable our progress in physical realms to catch up with our progress online.  Our journey requires applying a single unifying idea that is much simpler than attention - aim.   Precisely colliding a sequence of small, fast-moving gas or plasma balloons with a target vehicle delivers pulsed energy and momentum.   This enables a surprisingly broad range of groundbreaking applications.   Just as neural networks enabled consumer AI, they potentially allow us to achieve the precision control to make this external pulsed propulsion concept feasible.

\textbf{A Checklist of Grand Challenges External Pulsed Propulsion Can Solve}
\begin{itemize}
    \item 
Suborbital Transit:  We'll create a viable suborbital travel vehicle allowing passengers to take off from normal airport runways and reach anywhere in the world in less than 2 hours.   Noise pollution near population centers will be no worse than it is with conventional aircraft.
 \item Rocket Revolution: We'll end the tyranny of the rocket equation.   We'll still use small rockets, but giant rockets with high propellant mass fraction will no longer be needed to reach orbit.   
 
 \item Lunar Lift-off: Launching from the moon may still require volatiles, but not the more difficult task of making and storing high performance rocket fuel.
  \item Jurassic Dark: As a side effect of our advances, we'll create the field of lunar paleontology and discover concrete evidence for how life originated on earth and a genetic record of extinct species from ancient geological periods, like the dinosaurs. 
  \item Straw Ways to Heaven:   We can construct terrestrial megastructures, extending from the ground to the edge of space, without relying on advanced magnetic technologies such as Lofstrom Launch Loops \cite{lofstrom_loop}. A particularly ambitious yet beneficial example might be a vacuum tube connecting Earth and space, a "Straw Way to Heaven."
  \item Carbon Cancelled: We will solve our energy problems with carbon negative fuel that absorbs the carbon dioxide produced by industry, all while using minimal land and resources
  \item Moon mining:  We'll develop in-sutu resource utilization (ISRU) technology,first on our moon and then on icy moons like Saturn's moon Phoebe
\end{itemize}
Note, this paper summarizes some of the ideas in the blog ``Aim Is All You Need"\cite{aim2024}.
\end{abstract}

\section{Introduction}
Project Orion, conceived in the 1950s, remains the sole known propulsion technology offering both high specific impulse and high thrust \cite{projorion}. Its mechanism involved propelling a spacecraft by directing hypervelocity plasma from nuclear explosions onto a pusher plate. Despite its theoretical promise, Orion faced insurmountable challenges related to political feasibility, radioactive fallout, and the impractical mass requirements stemming from the high minimum yield of nuclear explosions. Nevertheless, Orion unequivocally validated the physics of hypervelocity pulsed propulsion.

Our approach replaces nuclear bombs with precisely aimed, hypervelocity gas balloons impacting a rocket's pusher plate. These balloons can be downscaled to practical sizes, and unlike nuclear explosions, small hypervelocity impacts could potentially be efficiently contained within a pulsed reaction chamber. This method would circumvent the restrictions of the Outer Space Treaty \cite{outer_space_treaty} and eliminate radioactive by-products.

Leveraging gravity assists, close solar approaches, and potentially in-situ resource mining, hypervelocity balloons could achieve energy densities and specific impulses comparable to Project Orion. This approach wouldn't demand nuclear or massive photovoltaic power, and it would enable lunar launches that use polar volatiles directly without needing to produce rocket fuel or electromagnetic mass drivers.     

The combination of recent neural network advancements and traditional control and localization techniques significantly enhances the plausibility of achieving the extremely precise navigation for externally pulsed propulsion and the radical applications it makes possible.



%Bibliography
\bibliographystyle{unsrt}  
\bibliography{references}  


\end{document}
